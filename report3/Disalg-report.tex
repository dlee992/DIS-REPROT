%!TEX TS-program = XeLaTeX
\documentclass[UTF8]{article}

%--
\usepackage{ctex}
\usepackage[margin=1in]{geometry}
\usepackage{amsmath}
\newtheorem{theorem}{Theorem}
\newtheorem{lemma}{Lemma}
\newtheorem{proof}{Proof}[section]

%--
\begin{document}
    
%--
{\flushleft \bf \Large 姓名:} 李达 (程序合成)
{\flushleft \bf \Large 旧学号:} DZ1633007
{\flushleft \bf \Large 新学号:} MG1633116
{\flushleft \bf \Large 日期:} 2021.01.10

%=========================================================================
\section*{论文信息}
\begin{itemize}
    \item Ongaro, D., \& Ousterhout, J. (2014). 
    In search of an understandable consensus algorithm. 
    In 2014 {USENIX} Annual Technical Conference ({USENIX}{ATC} 14) (pp. 305-319).
    \item Hunt, P., Konar, M., Junqueira, F. P., \& Reed, B. (2010, June). 
    ZooKeeper: Wait-free Coordination for Internet-scale Systems. 
    In USENIX annual technical conference (Vol. 8, No. 9).
\end{itemize}
    
%=========================================================================
\section{序言}
设计Raft的起因是在它之前,最常用且经典的共识算法是Paxos,虽然相关文献很多,但不管是业内专家还是对分布式感兴趣的普通人,都很难真正理解整个算法的正确性是如何保证的,
关键点在于缺乏直觉的把握和理解,那么实现起来就更加困难了。因此,Raft协议在设计之初,就明确考虑了可理解性,在完成共识算法该做的事情的同时,保证算法是直觉易于接受的,
同时也是实现起来模块化的,不同功能部分更加独立,不会有过多的耦合。

不过,在Raft出现之前,已经有很多工业级别的分布式服务系统出现了,比如ZooKeeper。ZooKeeper系统的设计初衷是提供一种分布式的基础服务,不同的分布式具体应用(也可叫做
Client端)可以按需使用这个基础服务提供的属性,来构建自己的上层分布式功能。ZooKeeper系统由Yahoo!公司研发和开源,已经在公司内部得到广泛的应用,作为搜索引擎的分布式
的基础服务,给多种上层应用作基础技术支撑。

Raft协议作为近十年来最经典的共识算法,ZooKeeper作为较为经典的提供分布式基础服务的工业级系统,都具备典型性,因此本次报告选择这两篇文章来分析和讨论。

%=========================================================================
\section{Raft}
Raft讨论的问题当然比ZooKeeper更加底层,考虑的是如何在多个servers之间达成共识。首先,Raft设计也是基于Replicated State Machine思想,通过记录操作日志,并对日志
进行多机备份,来防止server data diverge,保证所有读写操作之间的线性关系。这是一种很强的一致性要求,这种设计思路更多地考虑是确保Server作为一个整体是始终可用的,
读操作总能读到最近一次的写操作结果,不会读到陈旧值;同时也意味着不需要购买过多的Server来形成这样的整体,因为堆积Server数量,并不能线性地提升系统响应时间。

\subsection{基于Replicated State Machine的基本架构}
复制状态机模型看起来也很简单,构成状态机的每个Server内部有三个核心组成部分:共识模块,操作日志,和状态机本身。当外部Client发来读写请求时,首先经过共识模块的处理,
当大多数Server同意接受当前的Request时,就把该操作添加到日志的尾部,然后再更新状态机(比如一个KV列表)。在基本模型下,要考虑三个最关键的问题:如何进行Leader选举(
确保任何时候都最多只有一个Leader,确保系统启动时或少数Server宕机时,如何重新选举Leader),如何进行日志同步(如何确定对Leader日志加入新的内容,如何确定一个请求
是否被commit,如何同步宕机后重启的Server),如何确保安全(Client之间的写操作顺序得到保证,读操作确保不会读到陈旧的值)。这些都是任何一个共识算法需要应对的挑战。

\subsubsection{Leader Election}
Raft的Leader选举也描述地很简洁,首先给出了Server State的转换图。Server有三种状态:Follower,Candidate,和Leader。首先有个概念要先厘清,共识算法中总会有一个
Version,Term,Index的东西,用来记录当前选举的轮数。一般来讲,都是单调递增的,也就是说,如果一个Server是Leader,但通过通信发现,自身保存的Term比其他某个Server
保存的Term要小,就自动切入到Follower状态。

在系统启动或者Leader宕机的时候,每个Follower都会在停止接受到来自Leader的heartbeat一定时间范围后(在150-300ms之间,具体值随机确定),转入到Candidate状态,
并向其他Server发送投票请求,如果获得多数票,则当选为新一轮的Leader,并更新Term。如果选举时恰好出现平局,则大家都返回到Follower状态,等待超时重新发起选举,因为
Server本身数量不多,且超时时间长度具有随机性,很大概率确保很少出现平局重选的情况。

这就是Leader选举的基本过程。

\subsubsection{Log Replication}
Raft协议在日志同步方面,首先要确保做到不同Server的日志不会出现分叉。当然在特定时间点,允许不同Server的日志里记录未被多数确认的操作记录。而不出现分叉的定义,是以仅
考虑被多数提交的日志内容为前提,对于那么没有被提交却被记录在某个Server日志里的操作,会在后续和Leader同步的过程中被抹除,或者自己当选为Leader,就将这些尚未多数确认的
操作经过沟通变成提交过的日志,并改写与自身不同的Server日志。

结合Raft的Leader选举机制,在日志上还能确保如下属性:
\begin{enumerate}
    \item 如果两个Server各自日志中的两项具有相同的索引(日志计数)和Term(选举轮数),那么它们一定存储着相同的读或写操作内容。
    \item 如果两个Server各自日志中的两项具有相同的索引(日志计数)和Term(选举轮数),那么在这两项之前的所有日志是完全相同的。
\end{enumerate}

这两个性质是很强的一致性要求,当我们进行Leader和Follower之间日志同步时,我们只需要找到最近的一致项即可,之前的项根据属性可以确保相同,对后续项进行同步即可。


\subsubsection{Log Compaction}
日志压缩有一个行业内最简单的处理思路:打快照。问题也很明显,现实系统中,给日志的存储空间不是无限的,而且对于重新加入Server群体的机器来说,从头开始重新执行每个日志,也
不切实际。因此,在合适的时机,合适的机器上对当前的State Machine打快照,将当前数据保存到磁盘中,是一种非常合理的做法。既可以及时清理打快照之前的日志信息,对于新加入的
Server也可以先完整拷贝最新快照,再重新执行日志中的操作。

但打快照显然会遇到一个新问题:在快照的时候,是不是要暂定执行写操作。出于现实的考虑,Raft利用Copy-on-write的操作系统底层机制,当写操作发生时,对局部数据块进行单独拷贝,
这样在保证正确性的前提下,减少对性能的影响。另外一个值得考虑的问题:Leader当然可以执行打快照的操作,那么其他的Followers是否也可以在本地打快照呢?同样出于性能的考虑,
适当减轻Leader的读写负载,Raft也允许Followers打快照,当然前提是针对当前已经确定被commited的日志记录进行快照,对于尾部还未达成共识的日志,不能打到快照中去。


\subsection{耦合程度}
首先,Paxos算法分成四个步骤,几个步骤之间的耦合程度很高,无法脱离其他,单独理解其中一个阶段。但Raft算法耦合程度相对低,每个关键组件可以先单独理解,但同样有些部分不得不
进行耦合,因为单一一个步骤无法保证整个算法的安全性和正确性。这里举出一例:通过在Leader选举的过程中,确保被选举的Leader具有最长的已提交日志,这样省去在日志同步过程中,还要
考虑Follower向Leader输出日志项的情况,这正是一种耦合,不过正是因为这两者的相互作用关系,简化了在日志同步过程中的数据流,只有Leader向Follower输出日志,不会出现
反向输出。

%=========================================================================
\section{ZooKeeper}
ZooKeeper和Raft考虑的问题层面不同,ZooKeeper考虑的是:假定已经有了一种不错的共识算法,那么如何把具体的应用层需要和底层的分布式基础需求隔离开,也就是说,ZooKeeper
作为中间层,提供最基本的分布式原语,而基于原语,可以实现各种各样,功能丰富的分布式需求,比如通过基础原语,可以实现客户端之间的COnfiguration Management、Rendezvous、
Group Membership、Simple Locks、Read/Write Locks等等。在这些拓展的功能基础上,就可以进一步构建企业级的需求,比如Yahoo!的爬虫服务,发布-订阅系统等等。

\subsection{基本架构}
那提供什么样的原语,能够用于构建分布式应用呢?这样的原语能够达到哪种一致性强度?Chubby给出了一种设计思路和实际系统,而ZooKeeper给出了一个类似但不用锁的原语系统。

这套原语系统基于文件系统设计思路,通过目录和文件来管理不同的分布式应用,而客户端可以调用的核心原语API只有7个,主要包含创建、删除文件节点,判断文件是否存在、
读写文件数据、查询目录下的孩子集合、以及sync操作。

显然,不直接提供锁操作,为了实现分布式的锁操作,可以通过这些原语组合来实现。而ZooKeeper强大的一点在于,仅仅通过这7个原语,可以保证两个基本的读写一致性:
\begin{enumerate}
    \item 对于所有的写操作,确保线性,有全序;
    \item 对于单个客户端的读写操作,保证FIFO的顺序,既包含写操作,也包含读操作。
\end{enumerate}

言外之一,就是不保证不同客户端之间读操作的有序性,以及不保证不同客户端之间读操作和写操作的有序性。而之所以产生了这样的读写一致性,恰恰是应用层的需求带来的。
对于搜索引擎和很多分布式应用来说,往往单一时间段的读写请求数量是不均等的,read/write rate可能在10-100之间,那么如果可以放松对read一致性的要求,就有可能
带来服务器数量增长,响应延长接近线性增长的效果。这也是ZooKeeper和Raft在读写一致性上的区别。

\subsection{Wait-free}
ZooKeeper里反复提到了Wait-free的概念,对这个概念的理解我的想法是:仅仅在原语层不提供直接的锁或阻塞机制,但可以通过对原语的组合,来实现锁操作。同时在实际
实现ZooKeeper的系统内部,肯定用到了锁操作,来确保实现刚刚提到的读写一致性,但只是对内使用,对外不暴露这些功能。至于,这样做究竟有趣在哪来,我还需要继续思考。

%=========================================================================
\section{总结}
这段时间,看了更多的分布式应用的文章,看上去这是一个很有趣的领域,有很多奇妙的想法,用来实现各种各样的应用场景需求。今年毕业后,我也可能去工业界从事分布式应用的工作,
希望未来能对分布式系统,原理,架构,以及问题的本源有更多的理解和认识,以及深刻的实践。

\end{document}